\thispagestyle{empty}
\vspace{-0.7cm}

\cleanchapterquote{Aquí voy a poner frases épicas, una cosa que la gente lo vea y diga: pura coyeria.}{B. Riemann}{}

\section{Preliminares}

\lipsum[3]

\subsection{Introducción}
La función $\zeta$ de Riemann comienza su desarrollo histórico con Euler, recordemos que en 1735, Euler resuelve el problema de Basilea planteado por Pietro Mengoli y que los Bernoulli intentaron resolver sin éxito...

$$\sum_{n=1}^{\infty}\dfrac{1}{n^2}=\dfrac{\pi^2}{6}$$

6 años después Euler da una demostración que para la época podemos considerar formal, con dicho problema resuelto, se pregunta cómo generalizarlo y define lo que conocemos como la función $\zeta$.

\begin{definition}[La función $\zeta$]
Sea $s \in \mathbb{C}$ tal que $\Re(s)>1$, se define la función $\zeta$ como sigue:

$$\zeta(s)=\sum_{n=1}^{\infty}\dfrac{1}{n^s}$$
\end{definition}

\begin{note}
  \lipsum[1]
\end{note}


\subsection{El producto de Euler}
\lipsum[2]
\begin{theorem}
    Sea $s\in \mathbb{C}$ tal que $\Re(s)>1$, entonces:

    $$\zeta(s)=\sum_{n=1}^{\infty}\dfrac{1}{n^s}=\prod_p \dfrac{1}{1-p^{-s}}$$

    Donde el producto recorre todos los números primos.
\end{theorem}

Demos entonces la prueba de esto tal como la dió Euler\\

\begin{proof}
    Primero note que:

    $$(1-2^{-s})\zeta(s)=1+\dfrac{1}{3^s}+\dfrac{1}{5^s}+\dfrac{1}{7^s}+\dfrac{1}{9^s}+\cdots$$

    Recordemos que todo entero mayor que 1 es primo o producto de primos, de esto se sigue que...

    $$\zeta(s)\left(1-2^{-s}\right)(1-3^{-s})(1-5^{-s})\cdots(1-p_n^{-s})(1-p_{n+1}^{-s})\cdots=1$$

    Es decir...

    $$\zeta(s)=\prod_p (1-p^{-s})^{-1}$$
\end{proof}

\subsection{Algunas funciones aritmética}

\begin{definition}
    Sea $f$ una función, diremos que $f$ es una función aritmética si $f: \mathbb{N}\mapsto \mathbb{F}$, con $\mathbb{F}=\mathbb{R}$ o $\mathbb{F}=\mathbb{C}$, es decir, las funciones aritméticas son sucesiones de números reales o complejos.
\end{definition}


\begin{theorem}[TFA versión logarítmica]
    Sea $n\in \mathbb{N}$ tal que $n>1$:

    $$\log(n)=k_1\log(p_1)+k_2\log(p_2)+k_3\log(p_3)+\ldots+k_m\log(p_m)$$
\end{theorem}

Lo que sigue es un corolario

\begin{corollary}
    Sea $n\in \mathbb{N}$ y $n>1$ tenemos la siguiente propiedad:

    $$\log(n)=\sum_{j\mid n}\Lambda(j)$$
\end{corollary}

\lipsum[4]

\begin{eg}
    \lipsum[3]
\end{eg}

\lipsum[9]
\begin{lemma}
    Caja para  poner lemas \lipsum[4]
\end{lemma}

\lipsum[1-9]

\begin{prop}
    Proposición \lipsum[6]
\end{prop}