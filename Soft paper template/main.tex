\documentclass[12pt]{article}
\usepackage{float}
\usepackage{xcolor}
\usepackage{fourier,libertine}
\usepackage[T1]{fontenc}
\usepackage[spanish]{babel}
\usepackage{amsfonts,amsmath,amssymb,amsthm}
\usepackage{bbm}
\usepackage{geometry}
\setlength{\parindent}{0pt}
% Tcolorboxes
\makeatother
\usepackage{thmtools}
\usepackage[framemethod=TikZ]{mdframed}
\mdfsetup{skipabove=1em,skipbelow=1em}

\theoremstyle{definition}
\declaretheoremstyle[
    headfont=\bfseries\sffamily\color{black!70!black}, bodyfont=\normalfont,
    mdframed={
        linewidth=2pt,
        rightline=true, topline=false, bottomline=false,
        linecolor=black, backgroundcolor=black!3!white,
    }
]{thmbox}

\declaretheoremstyle[
    headfont=\bfseries\sffamily\color{black!3}, bodyfont=\normalfont,
    mdframed={
        linewidth=2pt,
        rightline=false, topline=false, bottomline=false,
        linecolor=black, backgroundcolor=black!3!white,align=center,
    }
]{thmtikz}


\declaretheoremstyle[
    headfont=\bfseries\sffamily\color{black!70!black}, bodyfont=\normalfont,
    numbered=no,
    mdframed={
        linewidth=0pt,
        rightline=false, topline=false, bottomline=false,
        linecolor=black, backgroundcolor=black!2!white,
    },
    qed=\qedsymbol
]{thmproofbox}

\declaretheorem[numberwithin=section,style=thmbox, name=Definición]{definition}
\declaretheorem[sibling=definition,style=thmbox, numbered=no, name=Ejemplo]{eg}
\declaretheorem[sibling=definition,style=thmbox, name=Proposición]{prop}
\declaretheorem[sibling=definition,style=thmbox, name=Teorema, numbered=yes]{theorem}
\declaretheorem[sibling=definition,style=thmbox, name=Lema]{lemma}
\declaretheorem[sibling=definition,style=thmbox, name=Corolario]{corollary}

\declaretheorem[style=thmproofbox, name=Demostración]{replacementproof}
\renewenvironment{proof}[1][\proofname]{\vspace{-10pt}\begin{replacementproof}}{\end{replacementproof}}

\declaretheorem[style=thmbox, numbered=no, name=Nota]{note}
\declaretheorem[style=thmbox, numbered=no, ]{temp}
\declaretheorem[style=thmtikz, numbered=no, name=.]{tikznt}

\newcommand{\bb}[1]{\mathbb{#1}}

\usepackage[most]{tcolorbox}
\tcbuselibrary{most}

\tcbset {
  base/.style={
    arc=7mm, 
    bottomtitle=0.5mm,
    boxrule=0mm,
    colbacktitle=black!90!white, 
    coltitle=white, 
    fonttitle=\bfseries, 
    left=2.5mm,
    leftrule=1mm,
    right=3.5mm,
    title={#1},
    toptitle=0.75mm, 
  }
}


\newtcolorbox{subbox}[1]{
  colframe=black!93!white,
  base={#1}
}
\usepackage{multicol}
\usepackage{blindtext}
\usepackage{lipsum}
 \geometry{
 a4paper,
 total={170mm,260mm},
 left=20mm,
 top=15mm,
 }
\title{\vspace{-2cm}\par\noindent\rule{16cm}{1pt}\large
\\\bfseries Título de lo que sea que se vaya a hacer en el trabajo
\vspace{-0.34cm}\par\noindent\hspace{0.15cm}\rule{16cm}{1pt}
\vspace{-0.2cm}
}
\author{\small \bfseries Logaritmo de pepe sobre pepe$.^1$\small \phantom{lkllkhhhhhhh}Mateo Andrés Manosalva Amaris$.^{2}$\\ \small \texttt{lpepe@unal.edu.co}\hspace{3.6cm}\texttt{mmanosalva@unal.edu.co}
}
\usepackage{titling}
\predate{\hspace{6.24cm}\small}
\postdate{}
%Atajos
\newcommand\N{\ensuremath{\mathbb{N}}}
\newcommand\R{\ensuremath{\mathbb{R}}}
\newcommand\Z{\ensuremath{\mathbb{Z}}}
\newcommand\Q{\ensuremath{\mathbb{Q}}}
\newcommand\C{\ensuremath{\mathbb{C}}}
\begin{document}
\maketitle
\begin{abstract}
\textcolor{red}{La idea de esto es una plantilla tipo paper que sea más soft a lo usual, es sobria pero no en exceso.}\lipsum[1]
\end{abstract}

\section{Preliminares}

\subsection{Introducción}
\lipsum[2]
\begin{definition}[La función $\zeta$]
Sea $s \in \mathbb{C}$ tal que $\Re(s)>1$, se define la función $\zeta$ como sigue:

$$\zeta(s)=\sum_{n=1}^{\infty}\dfrac{1}{n^s}$$
\end{definition}

\begin{note}
   \lipsum[3]
\end{note}

\lipsum[4]

\subsection{El producto de Euler}

\lipsum[3]

\begin{theorem}
    Sea $s\in \mathbb{C}$ tal que $\Re(s)>1$, entonces:

    $$\zeta(s)=\sum_{n=1}^{\infty}\dfrac{1}{n^s}=\prod_p \dfrac{1}{1-p^{-s}}$$

    Donde el producto recorre todos los números primos.
\end{theorem}


\begin{proof}

\textcolor{red}{Dejaré esto acá solo para que se vea un ejemplo de la fuente usada en esta plantilla}\\

    Primero note que:

    $$(1-2^{-s})\zeta(s)=1+\dfrac{1}{3^s}+\dfrac{1}{5^s}+\dfrac{1}{7^s}+\dfrac{1}{9^s}+\cdots$$

    Recordemos que todo entero mayor que 1 es primo o producto de primos, de esto se sigue que...

    $$\zeta(s)\left(1-2^{-s}\right)(1-3^{-s})(1-5^{-s})\cdots(1-p_n^{-s})(1-p_{n+1}^{-s})\cdots=1$$

    Es decir...

    $$\zeta(s)=\prod_p (1-p^{-s})^{-1}$$
\end{proof}

\begin{eg}
    \lipsum[1]
\end{eg}


\section{Algunas propiedades elementales en la distribución de los números primos.}

\subsection{La integral de Riemann-Stieltjes}

\lipsum[1] \cite{apostol1998introduction}

\begin{definition}
    \begin{itemize}\item[i)] Sean $P=\{x_0,x_1,...,x_n\}\in\mathcal{P}[a,b]$ (las particiones del intervalo $[a,b]$) y   $t_{k}\in [x_{k-1},x_k]$ cualesquiera. 
    
    Una suma de la forma $$S(P,f,\alpha)=\sum_{k=1}^{n}f(t_k)\Delta \alpha_k$$ se denomina \textbf{suma de Riemann-Stieltjes} de $f$ con respecto a $\alpha$ en $[a,b]$.
    
    \item[ii)] Decimos que $f$ es \textbf{Riemann-Integrable} con respecto a $\alpha$ en $[a,b]$, y escribimos ``$f\in \mathcal{R}(\alpha)$ en $[a,b]$'', si existe $A\in \mathbb R$ que satisface que:\\ 
    para todo $\epsilon>0$ existe $P_{\epsilon}\in \mathcal{P}[a,b]$ tal que si para toda $P\supset P_{\epsilon}$ y para cualquier elección de puntos $t_{k}\in [x_{k-1},x_k]$, entonces $$\mid S(P,f,\alpha) - A\mid <\epsilon.$$
    \end{itemize}
\end{definition}


\begin{prop}
    \lipsum[3]
\end{prop}


\begin{lemma}
    \lipsum[2]
\end{lemma}

\begin{center}
    \begin{tikzpicture}[scale=1.2]
\def\a{1.7}
\def\b{5.7}
\def\c{3.7}
\def\L{0.5} % width of interval

\pgfmathsetmacro{\Va}{2*sin(\a r+1)+4} \pgfmathresult
\pgfmathsetmacro{\Vb}{2*sin(\b r+1)+4} \pgfmathresult
\pgfmathsetmacro{\Vc}{2*sin(\c r+1)+4} \pgfmathresult

\draw[->,thick] (-0.5,0) -- (7,0) coordinate (x axis) node[below] {$x$};
\draw[->,thick] (0,-0.5) -- (0,7) coordinate (y axis) node[left] {$y$};
\foreach \f in {1.7,2.2,...,6.2} {\pgfmathparse{2*sin(\f r+1)+4} \pgfmathresult
\draw[fill=blue!20] (\f-\L/2,\pgfmathresult |- x axis) -- (\f-\L/2,\pgfmathresult) -- (\f+\L/2,\pgfmathresult) -- (\f+\L/2,\pgfmathresult |- x axis) -- cycle;}
\node at (\a-\L/2,-5pt) {\footnotesize{$a=x_0$}};
\node at (\b+\L/2+\L,-5pt) {\footnotesize{$b=x_n$}};
\draw[blue] (\c-\L/2,0) -- (\c-\L/2,\Vc) -- (\c+\L/2,\Vc) -- (\c+\L/2,0);
\draw[dashed] (\c,0) node[below] {\footnotesize{$\xi_i$}} -- (\c,\Vc) -- (0,\Vc) node[left] {$f(\xi_i)$};
\node at (\a+5*\L/2,-5pt) {\footnotesize{$x_{i-1}$}};
\node at (\a+7*\L/2,-5pt) {\footnotesize{$x_i$}};
\node at (\a+5*\L,-5pt) {\footnotesize{$x_{i+1}$}};
\draw[blue,thick,smooth,samples=100,domain=1.45:6.2] plot(\x,{2*sin(\x r+1)+4});
\filldraw[black] (\c,\Vc) circle (.03cm);
\end{tikzpicture}
\end{center}

\newpage
\bibliographystyle{unsrt}
\bibliography{references}
\nocite{*}




\end{document}


